\documentclass[10pt]{article}

\usepackage[utf8]{inputenc}
\usepackage[margin=0.5in]{geometry}
\usepackage{kpfonts}

\PassOptionsToPackage{usenames, dvipsnames}{xcolor}
\usepackage{figures/tikzit}

\usepackage{amsthm}
\usepackage{amssymb}
\usepackage{amsmath}
\usepackage{stmaryrd}
\usepackage{microtype}
\usepackage{enumitem}

\usepackage[style=alphabetic]{biblatex}
\usepackage[
    colorlinks=true,
    citecolor=Green, linkcolor=NavyBlue,
    urlcolor=BrickRed,
    pdftitle={A Fully Compositional Theory of Sequential Digital Circuits}
]{hyperref}
\usepackage[capitalise]{cleveref}

\input{macros/sets}
\input{macros/category}
\input{macros/streams}
\input{macros/circuits}
\input{macros/theorems}

\input{figures/circuits.tikzstyles}
\input{figures/circuits.tikzdefs}

\setcounter{biburlnumpenalty}{100}
\setcounter{biburlucpenalty}{100}
\setcounter{biburllcpenalty}{100}

\addbibresource{refs/refs.biblatex.bib}
\pagenumbering{gobble}

\title{
    \vspace{-3em}
    A Fully Compositional Theory of Sequential Digital Circuits
    \\
    \textbf{Extended abstract}
}
\author{\textbf{George Kaye}, David Sprunger and Dan R. Ghica}
\date{}

\begin{document}
\maketitle

\paragraph*{Contribution.}

Digital circuits are one of the defining technologies of the modern world, so it
may seem improbable that there are theoretical gaps remaining in their
understanding.
However, until recently we did not have a \emph{fully compositional} model of
digital circuits; by this we mean that a larger circuit can be built from
smaller circuits and interconnecting wires without paying heed to the internal
structure of these smaller circuits.
The sticking point was usually the presence of
\emph{non-delay-guarded feedback}~\cite{malik1994analysis}, which can lead to
undesired behaviour and is often forbidden in circuit design.
Nevertheless, to truly have a compositional theory for \emph{all} digital
circuits it cannot be ignored, and in fact careful use of it can lead to more
efficient circuits~\cite{riedel2004cyclic,riedel2012cyclic}.

This line of work was inspired by that of Lafont on
\emph{Boolean} circuits~~\cite{lafont2003algebraic}; our current work is the
direct successor to the more informal foundations of Ghica, Jung and Lopez, in
which digital circuits are modelled as morphisms in a symmetric traced
monoidal category (STMC)~\cite{ghica2016categorical,ghica2017diagrammatic}.
Our contributions are to make this work rigorous, and to this end we present
three sound and complete semantics for digital circuits: a brand new
\emph{denotational semantics} for digital circuits based on stream functions
with certain properties; a refinement and extension of the
\emph{operational semantics} presented in \cite{ghica2017diagrammatic} to
operate on open circuits; and a \emph{algebraic semantics} with which circuits
can be brought to a pseudo-normal form.

\paragraph*{Syntax.}

\paragraph*{Denotational semantics.}

\paragraph*{Operational semantics.}

\paragraph*{Algebraic semantics.}

\printbibliography[heading=bibintoc,title={References}]

\end{document}
