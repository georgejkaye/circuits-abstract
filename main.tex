\documentclass[10pt]{article}

\usepackage[utf8]{inputenc}
\usepackage[margin=0.5in]{geometry}
\usepackage{tgpagella}
\usepackage{mathpazo}

\PassOptionsToPackage{usenames, dvipsnames}{xcolor}
\usepackage{figures/tikzit}

\usepackage{amsthm}
\usepackage{amssymb}
\usepackage{amsmath}
\usepackage{stmaryrd}
\usepackage{microtype}
\usepackage{enumitem}
\usepackage{mathpartir}
\usepackage{mathtools}

\usepackage[style=alphabetic]{biblatex}
\usepackage[
    colorlinks=true,
    citecolor=Green,
    linkcolor=NavyBlue,
    urlcolor=BrickRed,
    pdftitle={A Fully Compositional Theory of Sequential Digital Circuits}
]{hyperref}
\usepackage[capitalise]{cleveref}

\input{macros/sets}
\input{macros/category}
\input{macros/streams}
\input{macros/circuits}
\input{macros/theorems}

\input{figures/circuits.tikzstyles}
\input{figures/circuits.tikzdefs}

\setcounter{biburlnumpenalty}{100}
\setcounter{biburlucpenalty}{100}
\setcounter{biburllcpenalty}{100}

\newcommand{\bgcolour}{white}

\addbibresource{refs/refs.biblatex.bib}
\pagenumbering{gobble}

\title{
    \vspace{-3em}
    A Fully Compositional Theory of Sequential Digital Circuits
    \\
    \textbf{Extended abstract}
}
\author{\textbf{George Kaye}, David Sprunger and Dan R. Ghica}
\date{}

\begin{document}
\maketitle

\paragraph*{Contribution.}

Digital circuits are ubiquitous, so it may seem improbable that there are
theoretical gaps remaining in their understanding.
However, until recently we did not have a \emph{fully compositional} model of
digital circuits; by this we mean that a larger circuit can be built from
smaller circuits without considering their internal structure.
The sticking point was usually the presence of
\emph{non-delay-guarded feedback}~\cite{malik1994analysis}, which can lead to
undesired behaviour and is often forbidden in circuit design.
Nevertheless, using it carefully can lead to more
efficient circuits~\cite{riedel2004cyclic,riedel2012cyclic}.

This line of work was inspired by that of Lafont on
\emph{Boolean} circuits~~\cite{lafont2003algebraic}; our current work is the
direct successor to the more informal foundations of Ghica, Jung and Lopez, in
which digital circuits are modelled as morphisms in a symmetric traced
monoidal category (STMC)~\cite{ghica2016categorical,ghica2017diagrammatic}.
Our contributions are to make this work rigorous, and to this end we present
three sound and complete semantics for digital circuits: a brand new
\emph{denotational semantics} for digital circuits based on stream functions
with certain properties; a refinement and extension of the
\emph{operational semantics} presented in \cite{ghica2017diagrammatic} to
operate on open circuits; and a \emph{algebraic semantics} with which circuits
can be brought to a pseudo-normal form.

\paragraph*{Syntax.}

In full generality, the category of sequential digital circuits is generated
over a \emph{circuit signature} specifying the components that make up circuits
and the signals that flow in the wires.
Here we will restrict to a concrete signature
representing \emph{gate-level circuits}; rather than Boolean values, we use the
four-valued system of \emph{Belnap logic}.

\begin{definition}[Belnap logic~\cite{belnap1977useful}]
    Let \(\values \coloneqq \{\bot,\belnapfalse,\belnaptrue,\top\}\) be the set
    of \emph{Belnap values}, respectively represent \emph{no signal}
    (a disconnected wire), a \emph{false signal}, a \emph{true signal}, and
    \emph{both signals at once} (a short circuit).
    These values form an \emph{information lattice} \((\values,\ljoin,\lmeet)\)
    and accept the usual operations of \(\land\), \(\lor\) and \(\neg\), both
    illustrated in \cref{fig:belnap}.
    Note that the operations are monotone.
\end{definition}

To aid in the presentation, we use the graphical calculus of
\emph{string diagrams}~\cite{joyal1991geometry,joyal1996traced,selinger2011survey},
in which terms equal by axioms of STMCs are depicted as isomorphism diagrams;
`only connectivity matters'.

\begin{definition}
    Let \(\scircsigma\) be the STMC freely generated over \(
    \iltikzfig{strings/structure/monoid/init}[colour=comb]
    \), \(
    \iltikzfig{circuits/components/values/vs}[val=\belnapfalse]
    \), \(
    \iltikzfig{circuits/components/values/vs}[val=\belnaptrue]
    \), \(
    \iltikzfig{circuits/components/values/vs}[val=\top]
    \), \(
    \iltikzfig{circuits/components/gates/and}
    \), \(
    \iltikzfig{circuits/components/gates/or}
    \), \(
    \iltikzfig{circuits/components/gates/not}
    \), \(
    \iltikzfig{strings/structure/comonoid/copy}[colour=comb]
    \), \(
    \iltikzfig{strings/structure/monoid/merge}[colour=comb]
    \), \(
    \iltikzfig{strings/structure/comonoid/discard}[colour=comb]
    \) and \(
    \iltikzfig{circuits/components/waveforms/delay}
    \).
\end{definition}

The first four generators are the values in \(\values\) where \(\bot\)
is denoted as \(\iltikzfig{strings/structure/monoid/init}[colour=comb]\).
These are followed by \(\andgate\), \(\orgate\) and \(\notgate\) gates, and
constructs for forking, joining and eliminating wires.
The final generator is a \emph{delay} of one unit of time.
Light blue circuits \(
\iltikzfig{strings/category/f}[box=f,colour=comb]
\)  are \emph{combinational}: they model functions.
Dark green circuits \(
\iltikzfig{strings/category/f}[box=f,colour=seq]
\) are \emph{sequential}: they have state.

\paragraph*{Denotational semantics.}

Circuits in \(\scircsigma\) are purely \emph{syntax}; we now interpret circuit
morphisms as \emph{stream functions}.
Streams \(\sigma,\tau \in \stream{A}\) can be viewed as functions
\(\nat \to A\), so we will use the notation \(\sigma(i)\) to obtain the \(i\)-th
element of a stream.

We must identify the properties of stream functions that describe circuit
behaviour.
They are \emph{causal}: the \(i\)-th element of a circuit can only be computed
from the first \(i\) elements of the inputs.
They are \emph{monotone} with respect to the pointwise ordering on stream
elements, as the components of circuits are monotone.
Finally they are \emph{finitely specified} as circuits have finitely many
states: although streams themselves are infinite, eventually a function to
compute the \(i\)-th element of an output stream based on the input stream must
repeat.

\begin{definition}
    Let \(\mathbf{Stream}\) be the PROP with morphisms \(m \to n\) the monotone,
    causal, finitely specified functions
    \(\valuetuplestream{m} \to \valuetuplestream{n}\).
\end{definition}

The first result we have is that this category is \emph{traced}, so it is a
suitable setting for modelling digital circuits with feedback.

\begin{proposition}[\cite{ghica2024fully}, Prop. 25]
    \(\mathbf{Stream}\) is traced.
\end{proposition}

Semantics are assigned using a PROP morphism
\(\morph{\circuittostream{}}{\scircsigma}{\mathbf{Stream}}\).
which acts on sequential components as \(
\circuittostream[
    \iltikzfig{circuits/components/values/vs}[val=v]
]{}()(0)
\coloneqq
v
\), \(
\circuittostream[
    \iltikzfig{circuits/components/values/vs}[val=v]
]{}()(k+1)
\coloneqq
\bot
\), \(
\circuittostream[
    \iltikzfig{circuits/components/waveforms/delay}
]{}(\sigma)(0)
\coloneqq
\bot
\), and \(
\circuittostream[
    \iltikzfig{circuits/components/waveforms/delay}
]{}(\sigma)(k+1)
\coloneqq
\sigma(k)
\).

\begin{figure}[p]
    \centering
    \tikzfig{circuits/a4}
    \qquad
    \begin{tabular}{|c|cccc|}
        \hline
        \(\land\)        & \(\bot\)         & \(\belnapfalse\) & \(\belnaptrue\)  & \(\top\)         \\
        \hline
        \(\bot\)         & \(\bot\)         & \(\belnapfalse\) & \(\bot\)         & \(\belnapfalse\) \\
        \(\belnapfalse\) & \(\belnapfalse\) & \(\belnapfalse\) & \(\belnapfalse\) & \(\belnapfalse\) \\
        \(\belnaptrue\)  & \(\bot\)         & \(\belnapfalse\) & \(\belnaptrue\)  & \(\top\)         \\
        \(\top\)         & \(\belnapfalse\) & \(\belnapfalse\) & \(\top\)         & \(\top\)         \\
        \hline
    \end{tabular}
    \quad
    \begin{tabular}{|c|c|}
        \hline
        \(\neg\)         &                  \\
        \hline
        \(\bot\)         & \(\bot\)         \\
        \(\belnaptrue\)  & \(\belnapfalse\) \\
        \(\belnapfalse\) & \(\belnaptrue\)  \\
        \(\top\)         & \(\top\)         \\
        \hline
    \end{tabular}
    \quad
    \begin{tabular}{|c|cccc|}
        \hline
        \(\lor\)         & \(\bot\)        & \(\belnapfalse\) & \(\belnaptrue\) & \(\top\)        \\
        \hline
        \(\bot\)         & \(\bot\)        & \(\bot\)         & \(\belnaptrue\) & \(\belnaptrue\) \\
        \(\belnapfalse\) & \(\bot\)        & \(\belnapfalse\) & \(\belnaptrue\) & \(\top\)        \\
        \(\belnaptrue\)  & \(\belnaptrue\) & \(\belnaptrue\)  & \(\belnaptrue\) & \(\belnaptrue\) \\
        \(\top\)         & \(\belnaptrue\) & \(\top\)         & \(\belnaptrue\) & \(\top\)        \\
        \hline
    \end{tabular}

    \caption{The lattice structure on \(\values\), and the truth tables
        of Belnap logic gates~\cite{belnap1977useful}.}
    \label{fig:belnap}
\end{figure}


\begin{definition}[Denotational equivalence]
    Two circuits are \emph{denotationally equivalent} \(
    \iltikzfig{strings/category/f}[box=f,colour=seq]
    \approx
    \iltikzfig{strings/category/f}[box=g,colour=seq]
    \) if \(
    \circuittostream[\iltikzfig{strings/category/f}[box=f,colour=seq]]{}
    =
    \circuittostream[\iltikzfig{strings/category/f}[box=g,colour=seq]]{}
    \).
\end{definition}

\begin{corollary}[\cite{ghica2024fully}, Cor. 67]
    There is an isomorphism of PROPs
    \(\scirc{\Sigma} / {\approx} \cong \mathbf{Stream}\).
\end{corollary}

\paragraph*{Operational semantics.}

To better relate the internal structure and behaviour of a circuit, we will now
define an \emph{operational semantics} which evaluates circuits using
\emph{reductions}.
For \(\listvar{v} \in \valuetuple{n}\), we write \(
\iltikzfig{circuits/components/waveforms/register-shorthand}[val=\listvar{v},width=n]
\coloneqq
\iltikzfig{circuits/components/waveforms/register}[val=\listvar{v},width=n]
\) for a `register'; we want to find the outputs of a circuit given some input
registers, i.e. reductions such that \
\(
\iltikzfig{circuits/productivity/productive-goal-lhs}
\reduction^\star
\iltikzfig{circuits/productivity/productive-goal-rhs}
\).

\begin{definition}
    A circuit is in \emph{pre-Mealy form} if it is in the form \(
    \iltikzfig{circuits/productivity/pre-mealy-form}[core=\hat{F},state=\listvar{s},dom=m,cod=n,trace=x,delay=y]
    \) and in \emph{Mealy form} if it is in the form \(
    \iltikzfig{circuits/productivity/mealy-form}[core=\hat{F},state=\listvar{s},dom=m,cod=n,delay=y]
    \).
\end{definition}

Any circuit can be translated into pre-Mealy form using the \((\mealyeqn)\)
rule in \cref{fig:rules}.
However, the \emph{non-delay-guarded feedback} blocks attempts to translate this
into Mealy form.
As the components of our circuits are monotone and we work in a finite lattice,
we can exploit the Kleene fixed-point theorem.

\begin{definition}[Iteration]
    For a combinational circuit \(
    \iltikzfig{strings/category/f-2-2}[box=f,colour=comb,dom1=x,dom2=m,cod1=x,cod2=n],
    \)
    let its \emph{\(n\)th iteration} \(
    \iltikzfig{strings/category/f-1-2}[box=f^n,colour=comb,dom=m,cod1=x,cod2=n]
    \) be defined inductively over \(n\) as \(
    \iltikzfig{circuits/instant-feedback/f0-box}[box=f,dom=m,trace=x,cod=n]
    \coloneqq
    \iltikzfig{circuits/instant-feedback/f0-definition}[box=f,dom=m,trace=x,cod=n]
    \) and \(
    \iltikzfig{circuits/instant-feedback/fkp1-box}[box=f,dom=m,trace=x,cod=n]
    \coloneqq
    \iltikzfig{circuits/instant-feedback/fkp1-definition}[box=f,dom=m,trace=x,cod=n]
    \).
\end{definition}

The \((\instantfeedbackeqn)\) rule in \cref{fig:rules} uses the fact that
the longest chain in \(\valuetuple{x}\) is \(2x+1\) to replace non-delay-guarded
feedback with a sequence of iterations; this means that any circuit can be
reduced to one in Mealy form.

To see how a circuit in Mealy form `processes' a value, we precompose it with
a register containing the inputs.
Applying the \((\streamingeqn)\) (`streaming') rule from \cref{fig:rules} to
this creates two copies  of the circuit: one for what is happening `now' and the
other for what is happening `later'.
The `now' circuit is some values applied to a combinational circuit: the four
rules on the right of \cref{fig:rules} can then be applied to reduce this to
an output value and a next state.

\begin{corollary}[Productivity (\cite{ghica2024fully}, Cor. 91)]\label{cor:productivity}
    For a sequential circuit \(
    \iltikzfig{strings/category/f}[box=f,colour=seq,dom=m,cod=n]
    \) and inputs \(\listvar{v} \in \valuetuple{m}\), there exists
    \(\listvar{w} \in \valuetuple{n}\) such that \(
    \iltikzfig{circuits/productivity/productive-goal-lhs}[box=f,input=\listvar{v},dom=m,cod=n]
    \reduction^\star
    \iltikzfig{circuits/productivity/productive-goal-rhs}[box=g,output=\listvar{w},dom=m,cod=n]
    \) by applying \(\mealyeqn\), \(\instantfeedbackeqn\) and \(\streamingeqn\)
    followed by the \((\forkeqn)\), \((\joineqn)\), \((\stubeqn)\)
    and the \((\gateeqn)\) rules exhaustively.
\end{corollary}

\begin{figure}[p]
    \centering
    \begin{mathpar}
        \inferrule{
            \iltikzfig{circuits/productivity/trace-delay}[core=f,state=\listvar{v},dom=m,cod=n,delay=y,trace=x,valwidth=z]
        }{
            \iltikzfig{circuits/productivity/mealy-rule}[core=f,trace=x,delays=y,values=z]
        }%
        \mealyeqn
        \quad
        \inferrule{
            \iltikzfig{circuits/instant-feedback/equation-lhs}[box=f]
        }{
            \iltikzfig{circuits/instant-feedback/fixpoint-concrete}[box=f]
        }
        \instantfeedbackeqn
        \quad
        \inferrule{
            \iltikzfig{circuits/axioms/generalised-streaming-lhs}[box=f]
        }{
            \iltikzfig{circuits/axioms/generalised-streaming-rhs}[box=f]
        }%
        \streamingeqn
        \quad
        \inferrule{
            \iltikzfig{circuits/axioms/fork-lhs}[val=v]
        }{
            \iltikzfig{circuits/axioms/fork-rhs}[val=v]
        }%
        \forkeqn
        \quad
        \inferrule{
            \iltikzfig{circuits/axioms/join-lhs}[val1=v,val2=w]
        }{
            \iltikzfig{circuits/axioms/join-rhs}[val1=v,val2=w]
        }%
        \joineqn
        \quad
        \inferrule{
            \iltikzfig{circuits/axioms/stub-lhs}[val=v]
        }{
            \iltikzfig{strings/monoidal/empty}
        }%
        \stubeqn
        \quad
        \inferrule{
            \iltikzfig{circuits/axioms/gate-lhs}[gate=p,input=\listvar{v}]
        }{
            \iltikzfig{circuits/axioms/gate-rhs}[gate=p,input=\listvar{v}]
        }%
        \gateeqn
    \end{mathpar}
    \caption{
        Productive reductions
    }
    \label{fig:rules}
\end{figure}

\begin{figure}[p]
    \begin{gather*}
        \iltikzfig{circuits/productivity/productive-goal-lhs}[box=f]
        \reduction
        \iltikzfig{circuits/productivity/mealy-form-applied}[core=\hat{f}]
        \reduction
        \iltikzfig{circuits/productivity/mealy-form-streamed}[core=\hat{f}]
        \reduction
        \iltikzfig{circuits/productivity/mealy-form-instant}[core=\hat{f}]
        \reduction
        \iltikzfig{circuits/productivity/mealy-form-instant-registers}[core=\hat{f}]
        \reduction
        \iltikzfig{circuits/productivity/mealy-form-produced}[core=\hat{f}]
        \coloneqq
        \iltikzfig{circuits/productivity/productive-goal-rhs}[box=G]
    \end{gather*}
    \caption{Processing inputs to a circuit}
    \label{fig:productivity}
\end{figure}

The general procedure for processing the inputs to a circuit is illustrated in
\cref{fig:productivity}.
This can be applied repeatedly in order to determine the outputs of a
circuit over time, and to tell if two circuits are \emph{observationally}
equivalent.

\begin{definition}[Observational equivalence]
    Two circuits with no more than \(c\) delay components are
    \emph{observationally equivalent} \(
    \iltikzfig{strings/category/f}[box=f,colour=seq]
    \sim
    \iltikzfig{strings/category/f}[box=g,colour=seq]
    \) if productivity produces the same outputs for all inputs of length
    \(|\values^c|+1\).
\end{definition}

\begin{corollary}[\cite{ghica2024fully}, Cor. 67]
    There is an isomorphism of PROPs
    \(\scirc{\Sigma} / {\sim} \cong \scirc{\Sigma} / {\approx}\).
\end{corollary}

\paragraph*{Algebraic semantics.}

The previous section gives an upper bound on checking observational equivalence,
but it is still an \emph{exponential} upper bound.
It is often more efficient to use \emph{equational reasoning} to determine if
two circuits have the same behaviour rather than fully evaluating their
behaviour.
An ad-hoc equational theory was presented in~\cite{ghica2016categorical} but
these were picked as `reasonable' equations on circuits; here we are guided by
the stream semantics.
The notion of Mealy form will once again come in useful, so we first reframe the
\((\mealyeqn)\) rule in terms of equations shown in \cref{fig:mealy-equations}.
The usual strategy to show that an equational theory is sound and complete is to
define some sort of `normal form' into which all terms can be translated.

\begin{figure}[p]
    \centering
    \(
    \equationdisplay{
        \iltikzfig{strings/structure/monoid/unitality-l-lhs}
    }{
        \iltikzfig{strings/structure/monoid/unitality-l-rhs}
    }{
        \monoidunitleqn
    }
    \quad
    \equationdisplay{
        \iltikzfig{strings/structure/monoid/unitality-r-lhs}
    }{
        \iltikzfig{strings/structure/monoid/unitality-r-rhs}
    }{
        \monoidunitreqn
    }
    \quad
    \equationdisplay{
        \iltikzfig{circuits/axioms/bottom-delay-lhs}
    }{
        \iltikzfig{circuits/axioms/bottom-delay-rhs}
    }{
        \bottomdelayeqn
    }
    \quad
    \equationdisplay{
        \iltikzfig{circuits/instant-feedback/equation-lhs}
    }{
        \iltikzfig{circuits/instant-feedback/equation-rhs}
    }{
        \instantfeedbackeqn
    }
    \)
    \caption{
        Set of Mealy equations
    }
    \label{fig:mealy-equations}
\end{figure}

\begin{definition}
    Let \(\lvert{-}\rvert\) be a fixed procedure mapping a function
    \(\morph{f}{\valuetuple{m}}{\valuetuple{n}}\) to a circuit \(
    \iltikzfig{strings/category/f}[box=|f|,colour=seq]
    \) such that \(
    \circuittostream[
        \iltikzfig{strings/category/f}[box=|f|,colour=seq]
    ]{}(\sigma)(i)
    =
    f(\sigma(i))
    \).
\end{definition}

One can think of \(|f|\) as the `canonical syntactic representation' of a
function; if a circuit is in this form it is easy to tell what function it
represents.
For Belnap circuits, this normal form is a variation of the standard disjunctive
normal form used for Booleans; the details are discussed
in~\cite[App. A]{ghica2024fully}.

\begin{lemma}
    Any combinational Belnap circuit \(
    \iltikzfig{strings/category/f}[box=f,colour=comb]
    \) is equal to a circuit in the image of \(\lvert{-}\rvert\) by the
    equations in \cref{fig:normal-form-equations}.
\end{lemma}

A circuit in Mealy form has a \emph{state} and a \emph{combinational core}.
To translate between two behaviourally equivalent circuits a good starting point
is to translate them into circuits with the same state.
We will omit the fine details, but intuitively, an
\emph{encoding function} is a monotone function
\(\morph{\gamma}{\valuetuple{x}}{\valuetuple{y}}\) which is `injective on the
states we care about', i.e.\ the next states produced by the combinational core
starting from some initial state.

\begin{lemma}
    Let \(\gamma\) be an encoding function with inverse \(\gamma^{-1}\); then
    the \emph{encoding equation} \((\mathsf{Encode})\) in
    \cref{fig:mealy-transformation-equations} is sound.
\end{lemma}

Even when two circuits in Mealy form have the same initial state, they may not
have behaviourally equivalent combinational cores; this is because the functions
are not equal but \emph{bisimilar} when starting from this initial state.
Given two functions we can compute if they are bisimilar from some initial state
by repeatedly executing them in `lockstep' for all inputs and checking that they
agree on the outputs for each successive pair of states; since there are
finitely many possible states this procedure will eventually terminate.

\begin{lemma}
    The family of \emph{bisimilarity equations} \((\mathsf{Bisim})\) in
    \cref{fig:mealy-transformation-equations} is sound.
\end{lemma}

We write \(\mathcal{E}\) for the equations in
\cref{fig:mealy-equations,fig:normal-form-equations,fig:mealy-transformation-equations},
which are enough for a sound and complete equational theory.
Given two denotationally equivalent circuits, we translate them into Mealy
form, use the encoding equation to translate them into Mealy form where the
initial state is a concatenation of the original states,
then use the bisimilarity equation.

\begin{theorem}
    There is an isomorphism of PROPs
    \(\scircsigma / \mathcal{E} \cong \scircsigma / {\approx}\).
\end{theorem}


\begin{figure}[p]
    \centering
    \begin{minipage}{0.3\textwidth}
        \centering
        \(\equationdisplay{
            \iltikzfig{circuits/axioms/belnap/translation/explosion-lhs}
        }{
            \iltikzfig{circuits/axioms/belnap/translation/explosion-rhs}
        }{
            \belnapexpeqn
        }\)
    \end{minipage}
    \begin{minipage}{0.3\textwidth}
        \centering
        \(\equationdisplay{
            \iltikzfig{circuits/axioms/belnap/translation/join-and-lhs}
        }{
            \iltikzfig{circuits/axioms/belnap/translation/join-and-rhs}
        }{
            \joinandeqn
        }\)
        \\[1em]
        \(\equationdisplay{
            \iltikzfig{circuits/axioms/belnap/translation/join-or-lhs}
        }{
            \iltikzfig{circuits/axioms/belnap/translation/join-or-rhs}
        }{
            \joinoreqn
        }\)
    \end{minipage}
    \begin{minipage}{0.3\textwidth}
        \centering
        \(\equationdisplay{
            \iltikzfig{circuits/axioms/belnap/translation/de-morgan-and-lhs}
        }{
            \iltikzfig{circuits/axioms/belnap/translation/de-morgan-and-rhs}
        }{
            \demorganand
        }\)
        \\
        \(\equationdisplay{
            \iltikzfig{circuits/axioms/belnap/translation/de-morgan-or-lhs}
        }{
            \iltikzfig{circuits/axioms/belnap/translation/de-morgan-or-rhs}
        }{
            \demorganor
        }\)
    \end{minipage}
    \\[1em]
    \(\equationdisplay{
        \iltikzfig{circuits/axioms/belnap/translation/not-fork-lhs}
    }{
        \iltikzfig{circuits/axioms/belnap/translation/not-fork-rhs}
    }{
        \notforkeqn
    }\)
    \quad
    \(\equationdisplay{
        \iltikzfig{circuits/axioms/belnap/translation/and-fork-lhs}
    }{
        \iltikzfig{circuits/axioms/belnap/translation/and-fork-rhs}
    }{
        \andforkeqn
    }\)
    \quad
    \(\equationdisplay{
        \iltikzfig{circuits/axioms/belnap/translation/or-fork-lhs}
    }{
        \iltikzfig{circuits/axioms/belnap/translation/or-fork-rhs}
    }{
        \orforkeqn
    }\)
    \quad
    \(\equationdisplay{
        \iltikzfig{strings/structure/bialgebra/merge-copy-lhs}
    }{
        \iltikzfig{strings/structure/bialgebra/merge-copy-rhs}
    }{
        \joinforkeqn
    }\)
    \\[1em]
    \(\equationdisplay{
        \iltikzfig{strings/structure/bialgebra/init-copy-lhs}
    }{
        \iltikzfig{strings/structure/bialgebra/init-copy-rhs}
    }{
        \botforkeqn
    }\)
    \quad
    \(\equationdisplay{
        \iltikzfig{circuits/axioms/belnap/translation/and-and-idempotent-lhs}
    }{
        \iltikzfig{circuits/axioms/belnap/translation/and-and-idempotent-rhs}
    }{
        \andandidemeqn
    }\)
    \quad
    \(\equationdisplay{
        \iltikzfig{circuits/axioms/belnap/translation/or-or-idempotent-lhs}
    }{
        \iltikzfig{circuits/axioms/belnap/translation/or-or-idempotent-rhs}
    }{
        \ororidemeqn
    }\)
    \\[1em]
    \quad
    \(\equationdisplay{
        \iltikzfig{circuits/axioms/belnap/and-or-distributivity-lhs}
    }{
        \iltikzfig{circuits/axioms/belnap/and-or-distributivity-rhs}
    }{
        \andorddisteqn
    }\)
    \quad
    \(\equationdisplay{
        \iltikzfig{circuits/axioms/belnap/or-and-distributivity-lhs}
    }{
        \iltikzfig{circuits/axioms/belnap/or-and-distributivity-rhs}
    }{
        \oranddisteqn
    }\)
    \quad
    \(\equationdisplay{
        \iltikzfig{strings/structure/comonoid/commutativity-lhs}
    }{
        \iltikzfig{strings/structure/comonoid/commutativity-rhs}
    }{
        \forkcommeqn
    }\)
    \\[1em]
    \(\equationdisplay{
        \iltikzfig{circuits/axioms/belnap/translation/bot-and-lhs}
    }{
        \iltikzfig{circuits/axioms/belnap/translation/bot-x-rhs}
    }{
        \botandeqn
    }\)
    \quad
    \(\equationdisplay{
        \iltikzfig{circuits/axioms/belnap/translation/bot-or-lhs}
    }{
        \iltikzfig{circuits/axioms/belnap/translation/bot-x-rhs}
    }{
        \botoreqn
    }\)
    \quad
    \(\equationdisplay{
        \iltikzfig{circuits/axioms/belnap/translation/bot-and-lhs}
    }{
        \iltikzfig{circuits/axioms/belnap/translation/bot-x-rhs}
    }{
        \botnoteqn
    }\)
    \quad
    \(\equationdisplay{
        \iltikzfig{circuits/axioms/belnap/double-negation-lhs}
    }{
        \iltikzfig{circuits/axioms/belnap/double-negation-rhs}
    }{
        \dneeqn
    }\)
    \\[1em]
    \(\equationdisplay{
        \iltikzfig{circuits/axioms/belnap/and-associativity-lhs}
    }{
        \iltikzfig{circuits/axioms/belnap/and-associativity-rhs}
    }{\andassoc}\)
    \quad
    \(\equationdisplay{
        \iltikzfig{circuits/axioms/belnap/or-associativity-lhs}
    }{
        \iltikzfig{circuits/axioms/belnap/or-associativity-rhs}
    }{\orassoc}\)
    \\[1em]
    \(\equationdisplay{
        \iltikzfig{circuits/axioms/belnap/and-commutativity-lhs}
    }{
        \iltikzfig{circuits/axioms/belnap/and-commutativity-rhs}
    }{\andcomm}\)
    \quad
    \(\equationdisplay{
        \iltikzfig{circuits/axioms/belnap/or-commutativity-lhs}
    }{
        \iltikzfig{circuits/axioms/belnap/or-commutativity-rhs}
    }{\orcomm}\)
    \quad
    \(\equationdisplay{
        \iltikzfig{circuits/axioms/belnap/and-idempotent-lhs}
    }{
        \iltikzfig{circuits/axioms/belnap/and-idempotent-rhs}
    }{\andidemeqn}\)
    \quad
    \(\equationdisplay{
        \iltikzfig{circuits/axioms/belnap/or-idempotent-lhs}
    }{
        \iltikzfig{circuits/axioms/belnap/or-idempotent-rhs}
    }{\oridemeqn}\)
    \\[1em]
    \(\equationdisplay{
        \iltikzfig{circuits/axioms/belnap/and-or-distributivity-lhs}
    }{
        \iltikzfig{circuits/axioms/belnap/and-or-distributivity-rhs}
    }{\andorddisteqn}\)
    \quad
    \(\equationdisplay{
        \iltikzfig{circuits/axioms/belnap/or-and-distributivity-lhs}
    }{
        \iltikzfig{circuits/axioms/belnap/or-and-distributivity-rhs}
    }{\oranddisteqn}\)
    \\[1em]
    \(\equationdisplay{
        \iltikzfig{circuits/axioms/belnap/absorption-2-lhs}
    }{
        \iltikzfig{circuits/axioms/belnap/absorption-2-rhs}
    }{\andabsoreqn}\)
    \quad
    \(\equationdisplay{
        \iltikzfig{circuits/axioms/belnap/absorption-1-lhs}
    }{
        \iltikzfig{circuits/axioms/belnap/absorption-1-rhs}
    }{\orabsandeqn}\)
    \quad
    \(\equationdisplay{
        \iltikzfig{circuits/axioms/belnap/translation/and-fork-lhs}
    }{
        \iltikzfig{circuits/axioms/belnap/translation/and-fork-rhs}
    }{\andforkeqn}\)
    \quad
    \(\equationdisplay{
        \iltikzfig{circuits/axioms/belnap/translation/or-fork-lhs}
    }{
        \iltikzfig{circuits/axioms/belnap/translation/or-fork-rhs}
    }{\andforkeqn}\)
    \\[1em]
    \(\equationdisplay{
        \iltikzfig{strings/structure/comonoid/commutativity-rhs}
    }{
        \iltikzfig{strings/structure/comonoid/commutativity-lhs}
    }{\forkcommeqn}\)
    \quad
    \(\equationdisplay{
        \iltikzfig{strings/structure/comonoid/associativity-lhs}
    }{
        \iltikzfig{strings/structure/comonoid/associativity-rhs}
    }{\forkassoceqn}\)
    \quad
    \(\equationdisplay{
        \iltikzfig{strings/structure/comonoid/unitality-l-lhs}
    }{
        \iltikzfig{strings/structure/comonoid/unitality-l-rhs}
    }{\forkuniteqn}\)
    \caption{
        Set of \emph{Belnap normalisation equations}
    }
    \label{fig:normal-form-equations}
\end{figure}

\begin{figure}
    \centering
    \(\equationdisplay{
        \iltikzfig{circuits/productivity/mealy-form}[core=|f|]
    }{
        \iltikzfig{circuits/algebraic/state-encoding}[core=|f|]
    }{\mathsf{Encode}}\)
    \qquad
    \(\equationdisplay{
        \iltikzfig{circuits/productivity/mealy-form}[core=|f|]
    }{
        \iltikzfig{circuits/productivity/mealy-form}[core=|g|]
    }{\mathsf{Bisim}}\)
    \,\,
    \begin{minipage}{0.15\textwidth}
        \centering
        where \(f\) and \(g\) are bisimilar from \(\listvar{s}\)
    \end{minipage}
    \caption{Families of \emph{Mealy transformation equations}}
    \label{fig:mealy-transformation-equations}
\end{figure}

\paragraph*{Conclusion.}

The fully compositional theory of sequential digital circuits sets a foundation
which will hopefully unlock new methods for working with circuits, such as
partial evaluation techniques.
One avenue we are particuarly interested in is that of \emph{automating}
circuit reasoning using \emph{string diagram graph rewriting}, which has been
studied in-depth recently~\cite{bonchi2022string,ghica2023rewriting}.

\printbibliography[heading=bibintoc,title={References}]

\end{document}
